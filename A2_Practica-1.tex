\documentclass[12pt,a4paper,oneside,onecolumn]{article}

\usepackage[spanish,es-noshorthands]{babel}
\usepackage{epsfig}
\usepackage[latin1]{inputenc}
\usepackage{amsmath}
\usepackage{amsfonts}
\usepackage{amssymb}
%\usepackage{mathabx}  Compila pero borra el pdf?
\usepackage{array}
\usepackage[left=1.8cm, right=1.8cm, top=2.50cm, bottom=2.5cm]{geometry}
\usepackage{hyperref}
\usepackage{color}
\usepackage{fancyhdr}
\usepackage{listings}
\usepackage{xcolor}

\pagestyle{fancy}

\fancyhead{}
\fancyfoot{}

\setlength{\headsep}{0.4cm}
\setlength{\footskip}{1.6pt}
\setlength{\parindent}{0pt}
\setlength{\extrarowheight}{1.5pt}


\lhead{An\'alisis Matem\'atico I}
\rhead{Javier Orti}
\renewcommand*\headrulewidth{0.4 pt}
\lfoot{\vspace{0.45cm}Pr\'actica 4}
\cfoot{\vspace{0.01cm}\rule{\linewidth}{0.4pt}}
\rfoot{\vspace{0.45cm} P\'ag. \thepage}

\renewcommand{\labelitemi}{$\bullet$}
\renewcommand{\labelenumi}{\theenumi)}
\renewcommand\spanishtablename{Tabla}

\decimalpoint

\headheight 16.7pt 
\textheight 715pt 

\parskip 8pt  

\hypersetup{
	colorlinks=true,
	linkcolor=blue,
	filecolor=magenta,      
	urlcolor=cyan,
}

% Python code settings
\definecolor{codegreen}{rgb}{0,0.6,0}
\definecolor{codegray}{rgb}{0.5,0.5,0.5}
\definecolor{codepurple}{rgb}{0.58,0,0.82}
\definecolor{backcolour}{rgb}{0.95,0.95,0.92}

\lstdefinestyle{mystyle}{
	backgroundcolor=\color{backcolour},   
	commentstyle=\color{codegreen},
	keywordstyle=\color{magenta},
	numberstyle=\tiny\color{codegray},
	stringstyle=\color{codepurple},
	basicstyle=\ttfamily\footnotesize,
	breakatwhitespace=false,         
	breaklines=true,                 
	captionpos=b,                    
	keepspaces=true,                 
	numbers=left,                    
	numbersep=5pt,                  
	showspaces=false,                
	showstringspaces=false,
	showtabs=false,                  
	tabsize=2
}

\lstset{style=mystyle}

\usepackage[shortlabels]{enumitem}
\usepackage{cancel}

\begin{document} 
     % Ejercicio 1 %
    \section{}
    a)
    Demostramos algebr\'aicamente:
    \[
        |\vec{x} \times \vec{y}|^2 = |\vec{x}|^2 \cdot |\vec{y}|^2 - (\vec{x} \cdot \vec{y})^2 
    \]
    \[
        \vec{x} = (a_1, a_2, a_3), \vec{y} = (b_1, b_2, b_3)
    \]
    \[
        \vec{x} \times \vec{y} = (a_2b_3 - a_3b_2, a_3b_1 - a_1b_3,a_1b_2 - a_2b_1)
    \]
    \[
        |\vec{x} \times \vec{y}|^2 = (a_2b_3 - a_3b_2)^2 + (a_3b_1 - a_1b_3)^2 + (a_1b_2 - a_2b_1)^2
    \]
    \[
        |\vec{x}|^2 = a_1^2 + a_2^2 + a_3^2,    
        |\vec{y}|^2 = b_1^2 + b_2^2 + b_3^2
    \]
    \[
        (\vec{x} \cdot \vec{y})^2 = (a_1b_1 +a_2b_2 + a_3b_3)^2
    \]
    Desarrollamos
    \[
        |\vec{x}|^2 \cdot |\vec{y}|^2 = a_1^2b_1^2 + a_2^2b_1^2 + a_3^2b_1^2 + a_1^2b_2^2 + a_2^2b_2^2 + a_3^2b_1^2 + a_3^2b_1^2 + a_3^2b_2^2 + a_3^2b_3^2 
    \]
    \[
        |\vec{x}|^2 \cdot |\vec{y}|^2 - (\vec{x} \cdot \vec{y})^2 = a_1^2b_1^2 + a_2^2b_1^2 + a_3^2b_1^2 + a_1^2b_2^2 + a_2^2b_2^2 + a_3^2b_1^2 + a_3^2b_1^2 + a_3^2b_2^2 + a_3^2b_3^2 -
    \]
    \[
        - (a_1b_1 +a_2b_2 + a_3b_3)^2 = 
    \]
    \[
        = a_2^2b_1^2 + a_3^2b_1^2 +a_1^2b_2^2 + a_3^2b_2^2 +a_1^2b_3^2 +a_2^2b_3^2 - 2a_1a_2b_2b_1 - 2a_1a_3b_3b_1 - 2a_2a_3b_2b_3
    \]
    Y comprobamos que est\'a tediosa expresi\'on coincide con $|\vec{x} \times \vec{y}|^2$:
    
    \[
        |\vec{x} \times \vec{y}|^2 = (a_2b_3 - a_3b_2)^2 + (a_3b_1 - a_1b_3)^2 + (a_1b_2 - a_2b_1)^2 =
    \]
    \[
        = a_2^2b_1^2 + a_3^2b_1^2 +a_1^2b_2^2 + a_3^2b_2^2 +a_1^2b_3^2 +a_2^2b_3^2 - 2a_1a_2b_2b_1 - 2a_1a_3b_3b_1 - 2a_2a_3b_2b_3
    \]
    Que si comprobamos t\'ermino a t\'ermino veremos que son iguales 
    
    b) Utilizando el resultado anterior, comprobamos:
    \[
        |\vec{x} \times \vec{y}| = |\vec{x}| \cdot |\vec{y}|\cdot |\sin{\theta}|
    \]
    Con la formula del coseno del angulo entre dos vectores, obtenemos que:
    \[
    |\vec{x}|^2 \cdot |\vec{y}|^2 - (\vec{x} \cdot \vec{y})^2 =  |\vec{x}|^2 \cdot |\vec{y}|^2 - (\vec{x}\vec{y}\cos{\theta})^2
    \]
    \[
        |\vec{x}|^2 \cdot |\vec{y}|^2 (1 - \cos^2{\theta})
    \]
    Y como sabemos por Pit\'agoras que $(1 - \cos^2{\theta}) = \sin^2{\theta}$ y aplicando una raiz, llegamos a la expresion inicial:
    \[
        \sqrt{|\vec{x}|^2 \cdot |\vec{y}|^2 \sin^2{\theta}}  = |\vec{x}| \cdot |\vec{y}|\cdot |\sin{\theta}| = |\vec{x} \times \vec{y}|
    \]
    
    % Ejercicio 2 %
    \section{}
    
    % Ejercicio 3 %
    \section{}
    
     % Ejercicio 4 %
     \section{}
     
	
\end{document}