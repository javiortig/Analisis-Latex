\documentclass[12pt,a4paper,oneside,onecolumn]{article}

\usepackage[spanish,es-noshorthands]{babel}
\usepackage{epsfig}
\usepackage[latin1]{inputenc}
\usepackage{amsmath}
\usepackage{amsfonts}
\usepackage{amssymb}
%\usepackage{mathabx}  Compila pero borra el pdf?
\usepackage{array}
\usepackage[left=1.8cm, right=1.8cm, top=2.50cm, bottom=2.5cm]{geometry}
\usepackage{hyperref}
\usepackage{color}
\usepackage{fancyhdr}
\usepackage{listings}
\usepackage{xcolor}

\pagestyle{fancy}

\fancyhead{}
\fancyfoot{}

\setlength{\headsep}{0.4cm}
\setlength{\footskip}{1.6pt}
\setlength{\parindent}{0pt}
\setlength{\extrarowheight}{1.5pt}


\lhead{An\'alisis Matem\'atico I}
\rhead{Javier Orti}
\renewcommand*\headrulewidth{0.4 pt}
\lfoot{\vspace{0.45cm}Pr\'actica 1}
\cfoot{\vspace{0.01cm}\rule{\linewidth}{0.4pt}}
\rfoot{\vspace{0.45cm} P\'ag. \thepage}

\renewcommand{\labelitemi}{$\bullet$}
\renewcommand{\labelenumi}{\theenumi)}
\renewcommand\spanishtablename{Tabla}

\decimalpoint

\headheight 16.7pt 
\textheight 715pt 

\parskip 8pt  

\hypersetup{
	colorlinks=true,
	linkcolor=blue,
	filecolor=magenta,      
	urlcolor=cyan,
}

% Python code settings
\definecolor{codegreen}{rgb}{0,0.6,0}
\definecolor{codegray}{rgb}{0.5,0.5,0.5}
\definecolor{codepurple}{rgb}{0.58,0,0.82}
\definecolor{backcolour}{rgb}{0.95,0.95,0.92}

\lstdefinestyle{mystyle}{
	backgroundcolor=\color{backcolour},   
	commentstyle=\color{codegreen},
	keywordstyle=\color{magenta},
	numberstyle=\tiny\color{codegray},
	stringstyle=\color{codepurple},
	basicstyle=\ttfamily\footnotesize,
	breakatwhitespace=false,         
	breaklines=true,                 
	captionpos=b,                    
	keepspaces=true,                 
	numbers=left,                    
	numbersep=5pt,                  
	showspaces=false,                
	showstringspaces=false,
	showtabs=false,                  
	tabsize=2
}

\lstset{style=mystyle}

\begin{document} 
	
	\begin{center}
		$\mbox{}$\\[2.0cm]
		\LARGE{\textbf{Software matem\'atico aplicado a las funciones de variable real. Problemas}}\\[2.0cm]
		
	\end{center}	
	
	\section{}
	
	\[
	\lim_{x \to \infty}(x -x^2\ln(1 + \frac{1}{x}))
	\]
	
	Este l\'imite puede ser resuelto sustituyendo \(x=1/x\), con lo que obtenemos: \[
	\lim_{y \to 0}\frac{y-ln(1+y)}{y^2}
	\]
	Y obten\'emos un l\'imite que puede ser resuelto con mayor facilidad:
	
	\vspace{0.3cm}
	\begin{figure}[!h]
		\centering
		\includegraphics[scale=0.25]{ex_1.png}
		\caption{}
		\label{fig:01}
	\end{figure}
	
	\section{}
	\[
	\lim_{x \to 0}\frac{x\sin{(\sin{(x)}})-(\sin(x))^2}{x^6}
	\]
	Obtenemos mediante \href{https://www.wolframalpha.com/input/?i=lim+x-%3E0+%28%28x*sen%28sen%28x%29%29+-+sen%5E2%28x%29%29%2F%28x%5E6%29%29}{\underline{WolframAlpha}} \(1/18\)
	\begin{figure}[!h]
		\centering
		\includegraphics[scale=0.35]{ex_2.png}
		\caption{}
		\label{fig:02}
	\end{figure}

	\section{}
	\[
	x^8+\frac{19}{2}x^7+\frac{355}{18}x^6+\frac{179}{18}x^5-\frac{59}{6}x^4-34x^3-\frac{710}{9}x^2-\frac{358}{9}x +\frac{70}{3} = 0
	\]
	Calculamos los \href{https://www.wolframalpha.com/input/?i=x%5E%288%29%2B%28%2819%29%2F%282%29%29+x%5E%287%29%2B%28%28355%29%2F%2818%29%29+x%5E%286%29%2B%28%28179%29%2F%2818%29%29+x%5E%285%29-%28%2859%29%2F%286%29%29+x%5E%284%29-34+x%5E%283%29-%28%28710%29%2F%289%29%29+x%5E%282%29-%28%28358%29%2F%289%29%29+x%2B%28%2870%29%2F%283%29%29}{\underline{cortes en el eje X}} tanto en el plano real como complejo
	\begin{figure}[!h]
		\centering
		\includegraphics[scale=0.4]{ex_3b.png}
		\caption{}
		\label{fig:03}
	\end{figure}

	\section{}
	Con el siguiente fragmento de c\'odigo:
	\lstinputlisting[language=Python, caption=\ ]{1_ex4.py}
	Obtenemos que: 
	\begin{figure}[!h]
		\centering
		\includegraphics[scale=0.35]{ex_4.png}
		\caption{}
		\label{fig:04}
	\end{figure}
	
	\section{}
	\[
	f(x) = x^6 - 8x^5 + 3x^3 - 16x^2 + 5
	\]
	Obtenemos los m\'aximos, m\'inimos y puntos de inflexi\'on mediante \href{https://www.geogebra.org/calculator}{\underline{Geogebra}}:
	\newline \newline \newline \newline \newline \newline \newline \newline \newline 
	
	\begin{figure}[!h]
		\centering
		\includegraphics[scale=0.3]{ex_5.png}
		\caption{}
		\label{fig:05}
	\end{figure}

	\section{}
	Calculamos gr\'aficamente las as\'intotas:
	\[
		f(x) = \frac{x^3 - x^2 + 1}{x^4 - x^3 - x^2 + 1}
	\]
	Para las asíntotas verticales, igualamos el denominador a 0 obteniendo \(x = 1\) e \(x = 1.3247\).
	\newline Al representar gr\'aficamente la funci\'on, comprobamos que tiene una as\'intota vertical en \(y = 0\)
		 
	
	\begin{figure}[!h]
		\centering
		\includegraphics[scale=0.22]{ex_6b.png}
		\caption{}
		\label{fig:06}
	\end{figure}
	
	\section{}
	\[
		f(x) = \cos{(x)} + 2\sin{(x)}\text{, with } c = \pi/4
	\]
	Con el siguiente fragmento de c\'odigo:
	\lstinputlisting[language=Python, caption=\ ]{1_ex7.py}
	Y el output:
	\begin{figure}[!h]
		\centering
		\includegraphics[scale=0.33]{ex_7.png}
		\caption{}
		\label{fig:07}
	\end{figure}

	\section{}
	Resolvemos la integral mediante el siguiente fragmento con la librer\'ia \textit{sympy}
	\[
		\int_{0}^{\pi/4}{\ln{(\cos{(x)})}dx} = -0.0864137
	\]
	\lstinputlisting[language=Python, caption=\ ]{1_ex8.py}
	
	\section{}
	resolvemos las siguientes integrales:
	\[
	\int_{2}^{10}{(\frac{1}{x} - \frac{3x^2 - 20}{3x^2 + 6x + 10})} = -2.25954
	\]
	\[
	\int_{2}^{10}{|\frac{1}{x} - \frac{3x^2 - 20}{3x^2 + 6x + 10}|} = 3.26821
	\]
	\begin{figure}[!h]
		\centering
		\includegraphics[scale=0.4]{ex_9a.png}
		\caption{}
		\label{fig:08}
	\end{figure}
	\begin{figure}[!h]
		\centering
		\includegraphics[scale=0.6]{ex_9b.png}
		\caption{}
		\label{fig:09}
	\end{figure}

	\section{}
	Determinamos el arco mediante la ecuaci\'on 
	\[
		L = \int_{a}^{b}\sqrt{1 + (f'(x))^2}
	\]
	y obtenemos con la calculadora gr\'afica:
	\begin{figure}[!h]
		\centering
		\includegraphics[scale=0.35]{ex_10.png}
		\caption{}
		\label{fig:10}
	\end{figure}
	\newline \newline 
	y que la longitud es \(L = 3.0422965\)
\end{document}


